% !TEX TS-program = xelatex
% !TEX encoding = UTF-8 Unicode
% !Mode:: "TeX:UTF-8"

\documentclass{resume}
\usepackage{zh_CN-Adobefonts_external} % Simplified Chinese Support using external fonts (./fonts/zh_CN-Adobe/)
%\usepackage{zh_CN-Adobefonts_internal} % Simplified Chinese Support using system fonts
\usepackage{linespacing_fix} % disable extra space before next section
\usepackage{cite}
\usepackage{hyperref}

\begin{document}
\pagenumbering{gobble} % suppress displaying page number

\name{李振源}

\contactInfo{(+86) 183-9259-3905}{lizhenyuan@zju.edu.cn}{浙江大学 网络空间安全 博士}{GitHub @li-zhenyuan}


% \section{个人总结}
% \textbf{}


\section{教育背景}
\datedsubsection{\textbf{浙江大学},网络空间安全,\textit{在读博士研究生}}{2017.9 - 2022.6}
浙江大学博士生学术新星项目(CNY 112K), CSC公派留学生联培博士项目, 三好研究生, 优秀博士生岗位助学金(CNY 10K), 预计2022年6月毕业
\datedsubsection{\textbf{西安电子科技大学},信息安全,\textit{工学学士}}{2013.9 - 2017.6}
排名1/154(前1\%),国家奖学金,西安电子科技大学优秀毕业生标兵(~20/5000),科技竞赛奖(2次),优秀学生标兵/优秀学生(3次)


\section{项目经历}
\datedsubsection{\textbf{基于系统级溯源图的入侵检测与分析} {[2,3]}}{2019.8 - present}
\begin{itemize}
  \item 随着信息技术的发展, 网络空间的边界不断延展, 向攻击者暴露了越来越多的漏洞. 传统的补漏式的防御策略越来越难以为继. 安全从业人员和研究者亟需新的工具来描述攻击并进行检测. 系统级的溯源图能够将系统中的行为根据因果关系联系起来, 很适合检测现在常见的多阶段的高级持续性攻击. 因此被广泛的研究. 
  \item 我们在已有工作的基础上, 主要研究普遍出现的依赖爆炸带来的误报和效率问题. 尝试设计基于日志流的, 轻量, 准确, 低延迟的检测系统。 
  \item 关键词: Provenance Graph, Threat Detection, Causality Analysis, Dependence Explosion Problem.
\end{itemize}

\datedsubsection{\textbf{高效且轻量的 PowerShell 解混淆框架} {[1,5]}}{2018.7 - 2019.11}
\begin{itemize}
  \item 近年来, PowerShell 被广泛的用于多种网络攻击, 包括钓鱼邮件攻击, 勒索病毒, 无文件攻击等. 但是由于 PowerShell 的动态特性, 可很方便的混淆和实施无文件攻击. 因此现有的检测方法无法准确的检测基于 PowerShell 的攻击. 
  \item 我们针对关键的混淆问题, 设计了一套细粒度(抽象语法子树)的, 准确且轻量的解混淆系统. 实验证明可以有效处理多种, 多层的混淆, 且解混淆可以有效提高检测的准确度. 
  \item 关键词: PowerShell (Scripting Language), Fileless Attack, De-Obfuscation, AST.
\end{itemize}

\datedsubsection{\textbf{终端安全和高级持续性威胁 (APT) 的检测} {[4]}}{2016.7 - 2020.10}
\begin{itemize}
  \item 高级持续性威胁可以对政府和企业的高价值目标造成严重的威胁. 受到了业界和学术界的广泛关注. 但是现有的检测方法既不够细粒度也不鲁棒, 无法准确的检测 APT 攻击.
  \item 本项目主要研究了被广泛用于 APT 攻击的远控木马 (RAT) 的多种攻击行为。研发了实时, 细粒度的检测系统 {\it APTShield} 通过先检测可疑的行为, 然后结合上下文信息对初步检测结果进行过滤.
  \item 关键词: APT, RAT, Event Tracing for Windows (ETW), Potential Harmful Behavior
\end{itemize} 


\section{论文列表}
% increase linespacing [parsep=0.5ex]
\begin{itemize}[parsep=0.2ex]
\item {[1]} {\bf Zhenyuan Li}, Qi Alfred Chen, Chunlin Xiong, Yan Chen, Tiantian Zhu,and Hai Yang. ``Effective and Light-Weight Deobfuscation and Semantic-Aware Attack Detection for PowerShell Scripts'', In ACM Conference on Computer and Communications Security 2019 ({\bf ACM CCS'19}). \href{https://dl.acm.org/doi/10.1145/3319535.3363187}{[Paper, Vedio]} \href{https://github.com/li-zhenyuan/PowerShellDeobfuscation}{[Code]}
\item {[2]} {\bf Zhenyuan Li}, Qi Alfred Chen, Yang Runqing, Yan Chen. ``Threat Detection and Investigation with System-level Provenance Graphs: A Survey'', (arXiv:2006.01722v1 [cs.CR]). ArXiv Computer Science. (2020). \href{https://arxiv.org/pdf/2006.01722}{[Paper]} (Revision, {Computer \& Security})
\item {[3]} {\bf Zhenyuan Li}, Yang Runqing, Qi Alfred Chen, Yan Chen. ``A First Look at Evasion against Provenance Graph-based Threat Detection''. (Under submission, ACSAC' 20 Poster session)
\item {[4]} Runqing Yang, Xutong Chen, Haitao Xu, Yueqiang Chen, Chunlin Xiong, Linqi Ruan, Mohammad Kavousl, {\bf Zhenyuan Li}, Liheng Xu, Yan Chen. ``RATScope: Recording and Reconstructing Missing RAT Semantic Behaviors for Forensic Analysis on Windows'', to appear in IEEE Transactions on Dependable and Secure Computing ({\bf IEEE TDSC' 21}) \href{https://doi.org/10.1109/TDSC.2020.3032570}{[Paper]}
\item {[5]} Chunlin Xiong, {\bf Zhenyuan Li}, Qi Alfred Chen, Yan Chen, Tiantian Zhu, Hai Yang, and Wei Ruan. ``Generic, Efficient, and Effective Deobfuscation and Semantic-Aware Attack Detection for PowerShell Scripts''(Revision, FITEE)
\end{itemize}


\section{学术活动}
\begin{itemize}[parsep=0.2ex]
  \item 于 ``ACM CCS 2019'' 做报告展示关于 PowerShell 解混淆的最新工作, 英国伦敦
  \item 受邀于 ``InforSec 网络空间安全国际顶会论文分享会做报告'' (北京网络空间安全大会分会场), 在线会议
  \item Reviewer: IEEE ACCESS (2020)
  \item SubReviewer: CCS'19, ICDCS'19, ESORICS'19, CCS'18
\end{itemize}

\end{document}
